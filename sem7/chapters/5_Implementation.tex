\chapter{Implementation}

\section{Codebase Structure}
The repository follows a conventional Next.js 14 App Router layout with typed modules and Redux state. A simplified view:\cite{nextjs,rtk}
\begin{verbatim}
DevForge/
├── app/
│   ├── api/
│   │   ├── auth/           # NextAuth routes
│   │   ├── projects/       # CRUD project routes
│   │   ├── files/          # read/write/tree
│   │   └── docker/         # container lifecycle, logs
│   ├── auth/               # signin/signup UI
│   ├── dashboard/          # projects list
│   └── ide/[projectId]/    # IDE shell: files | editor | terminal
├── components/ide/         # FileExplorer, MonacoEditor, Terminal
├── lib/{auth,mongodb,redis,docker}/
├── store/{slices,thunks,middleware}/
└── types/                  # shared types
\end{verbatim}

\section{Key Modules}
\subsection{Authentication}
NextAuth credential provider with bcrypt hashing and JWT-based sessions protects API routes and pages.\cite{nextauth,bcryptjs}

\subsection{Database Access}
MongoDB stores users and projects. Connections are pooled via a shared client; models define indices for frequent lookups.\cite{mongodb}

\subsection{Redis Port Allocation}
Ports are drawn from configured ranges and reserved atomically to avoid conflicts.\cite{redis}

\subsection{Docker Orchestration}
Containers are provisioned with mounted volumes and restart policy \texttt{unless-stopped}. dockerode is used to create/start/stop and fetch logs.\cite{docker,dockerode}

\subsection{Editor and Files}
The Monaco editor auto-saves with a debounce; Files API serves directory trees and read/write operations.\cite{monaco}

\section{Representative Snippets}
\subsection{Creating a Container (dockerode)}
\begin{verbatim}
import Docker from 'dockerode'

const docker = new Docker()
const container = await docker.createContainer({
    Image: 'node:18-alpine',
    name: `devforge-${projectId}`,
    HostConfig: {
        Binds: [`${hostPath}:${containerPath}`],
        RestartPolicy: { Name: 'unless-stopped' },
        Memory: 512 * 1024 * 1024,
        NanoCPUs: 1e9,
    },
    Env: [
        `PORT=${assignedPort}`
    ],
    ExposedPorts: { [`${assignedPort}/tcp`]: {} },
    Cmd: ['sh', '-lc', 'npm install && npm run dev']
})
await container.start()
\end{verbatim}

\subsection{Files: Tree Endpoint}
\begin{verbatim}
export async function GET(req) {
    const dir = new URL(req.url).searchParams.get('dir')
    const nodes = await readTree(dir)
    return NextResponse.json({ nodes })
}
\end{verbatim}

\subsection{Editor Auto-save (Redux Thunk)}
\begin{verbatim}
export const saveFile = createAsyncThunk('files/save', async ({ path, content }) => {
    await fetch('/api/files/write', { method: 'POST', body: JSON.stringify({ path, content }) })
})
\end{verbatim}

\section{Security Considerations}
\begin{itemize}
    \item Hash passwords with bcrypt; never store plaintext.\cite{bcryptjs}
    \item Validate file paths to prevent directory traversal.
    \item Scope container volumes to project directories only.
    \item Limit CPU/RAM per container; avoid privileged mode.
\end{itemize}

\section{Deployment Notes}
DevForge runs on macOS/Linux with Docker Desktop/Engine. Environment variables configure MongoDB, Redis, and JWT secrets. Reverse proxying (Nginx/Caddy) can terminate TLS in production.

