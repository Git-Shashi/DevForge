\chapter{Implementation}
\section{Code Structure}

The code for the F1 Simulation Project is structured into multiple files to ensure clarity and organization. Below is a brief overview of the code files:

\begin{itemize}
    \item \texttt{f1\_simulation.h}: This header file contains the definitions for structures, constants, and function declarations used in the simulation.
    \item \texttt{f1\_functions.c}: This file includes the implementation of functions that facilitate the simulation, such as displaying information, choosing teams and drivers, and simulating the race.
    \item \texttt{f1\_simulation.c}: This is the main file that integrates all functionalities, executes the simulation, and displays results.
\end{itemize}

\section{Code Implementation}

\subsection{Header File: f1\_simulation.h}

The header file includes necessary libraries, defines constants, and declares structures and functions.

\begin{verbatim}
#ifndef F1_SIMULATION_H
#define F1_SIMULATION_H

#include <stdio.h>
#include <stdlib.h>
#include <string.h>
#include <time.h>

#define MAX_NAME_LENGTH 50
#define NUM_TEAMS 5
#define NUM_VENUES 10
#define NUM_WEATHER_CONDITIONS 4

typedef struct {
    char name[MAX_NAME_LENGTH];
    char team[MAX_NAME_LENGTH];
    int horsepower;
    int reliability;
    int aerodynamics;
} Driver;

typedef struct {
    char name[MAX_NAME_LENGTH];
    float length;  // in kilometers
    int difficulty;  // 1-10 scale
} Venue;

typedef enum {
    SUNNY,
    CLOUDY,
    RAINY,
    STORMY
} Weather;

// Global variables declarations
extern const char *team_names[NUM_TEAMS];
extern const char *driver_names[NUM_TEAMS][2];
extern Venue venues[NUM_VENUES];

// Function declarations
void display_f1_info();
int choose_team();
void choose_driver(Driver *driver, int team_index);
void input_car_specs(Driver *driver);
int choose_venue(Venue venues[]);
Weather choose_weather();
int input_laps();
void simulate_race(Driver drivers[], int num_drivers, Venue venue, Weather weather, int laps);
void display_podium(Driver drivers[], int num_drivers);

#endif // F1_SIMULATION_H
\end{verbatim}

\subsection{Functions Implementation: f1\_functions.c}

This file implements the various functions declared in the header file. Here is a snippet of the \texttt{display\_f1\_info} function:

\begin{verbatim}
#include "f1_simulation.h"

void display_f1_info() {
    printf("Welcome to the F1 Racing Simulation!\n\n");
    printf("Formula 1 (F1) is the highest class of international auto racing for single-seater formula racing cars.\n");
    printf("F1 seasons consist of a series of races, known as Grands Prix, held worldwide on purpose-built circuits and public roads.\n");
    printf("The results of each race are evaluated using a points system to determine two annual World Championships: one for drivers, the other for constructors.\n\n");
}
\end{verbatim}

\subsection{Main Implementation: f1\_simulation.c}

The main file integrates all functionalities and executes the simulation. Below is the main function which serves as the entry point for the program.

\begin{verbatim}
#include "f1_simulation.h"

// Global variables definitions
const char *team_names[NUM_TEAMS] = {
    "Mercedes", "Red Bull Racing", "Ferrari", "McLaren", "Aston Martin"
};

const char *driver_names[NUM_TEAMS][2] = {
    {"Lewis Hamilton", "George Russell"},
    {"Max Verstappen", "Sergio Perez"},
    {"Charles Leclerc", "Carlos Sainz"},
    {"Lando Norris", "Oscar Piastri"},
    {"Fernando Alonso", "Lance Stroll"}
};

Venue venues[NUM_VENUES] = {
    {"Monaco", 3.337, 10},
    {"Silverstone", 5.891, 8},
    {"Monza", 5.793, 7},
    {"Spa-Francorchamps", 7.004, 9},
    {"Suzuka", 5.807, 8},
    {"Interlagos", 4.309, 7},
    {"Melbourne", 5.303, 6},
    {"Singapore", 5.063, 9},
    {"Baku", 6.003, 8},
    {"Montreal", 4.361, 7}
};

int main() {
    srand(time(NULL));
    Driver drivers[NUM_TEAMS];
    int chosen_venue;
    Weather weather;
    int laps;

    display_f1_info();

    // Let the user choose all five teams and drivers
    for (int i = 0; i < NUM_TEAMS; i++) {
        printf("\nChoosing for Team %d:\n", i + 1);
        int chosen_team = choose_team();
        choose_driver(&drivers[i], chosen_team);
        input_car_specs(&drivers[i]);
    }

    chosen_venue = choose_venue(venues);
    weather = choose_weather();
    laps = input_laps();

    simulate_race(drivers, NUM_TEAMS, venues[chosen_venue], weather, laps);
    display_podium(drivers, NUM_TEAMS);

    return 0;
}
\end{verbatim}

\subsection{Output}

% Paste your output image here
\begin{figure}[h!]
    \centering
    \includegraphics[width=\textwidth]{path/to/your/output_image.png} % Adjust the path as necessary
    \caption{Race Simulation Output}
    \label{fig:simulation_output}
\end{figure}

\section{Results and Observations}

During the testing phase, various scenarios were simulated to assess the functionality and performance of the code. The results varied based on team selections, driver performance, and environmental conditions.

\subsection{Key Observations}

\begin{itemize}
    \item The simulation successfully evaluates driver performance based on their car specifications.
    \item Race outcomes were influenced by factors such as weather conditions and venue difficulty.
    \item The podium display effectively showcased the race results along with commentary, enhancing the user experience.
\end{itemize}

\end{document}

