\chapter{Requirement Engineering}

\section{Introduction}
Requirement engineering is a crucial phase in software development, where user needs are gathered, analyzed, and translated into functional and non-functional requirements. This chapter outlines the requirements for the F1 Simulation Project based on the user interactions, system functionality, and performance objectives. The goal is to ensure that the system aligns with user expectations and provides an intuitive, responsive race simulation experience.

\section{User Requirements}
The primary users of the F1 Simulation Project are motorsport enthusiasts who wish to simulate Formula 1 races, and the secondary users are evaluators who analyze the simulation results. The user requirements are as follows:
\begin{itemize}
    \item \textbf{Easy-to-use interface:} Users should be able to easily select cars, drivers, and venues without complicated navigation.
    \item \textbf{Customizable race parameters:} Users should be able to input performance values such as horsepower, reliability, and aerodynamics.
    \item \textbf{Detailed race results:} Users expect detailed output, including lap-by-lap statistics, podium standings, and race commentary.
    \item \textbf{Realistic performance calculations:} The system must calculate race results based on realistic factors such as horsepower and venue characteristics.
\end{itemize}

\section{Functional Requirements}
The functional requirements define the core functionality that the F1 Simulation Project must deliver:
\begin{itemize}
    \item \textbf{Race Simulation:} The system must simulate a Formula 1 race, considering user-selected parameters such as car manufacturer, driver, venue, and number of laps.
    \item \textbf{Statistics Tracker:} The system must track driver statistics, including lap times, race positions, and overall performance.
    \item \textbf{Podium Standings Display:} The system must calculate and display podium standings based on race outcomes.
    \item \textbf{Commentary Generator:} The system must generate race commentary based on the car manufacturer and driver performance.
    \item \textbf{Input Interface:} The system must provide a user-friendly interface where users can select options from dropdown menus for car manufacturers, drivers, and venues.
\end{itemize}

\section{Non-functional Requirements}
Non-functional requirements specify the performance and usability aspects of the system:
\begin{itemize}
    \item \textbf{Performance:} The system must calculate race results within a few seconds after input submission to maintain user engagement.
    \item \textbf{Scalability:} The system should be designed to handle additional features or user inputs in future versions.
    \item \textbf{Reliability:} The system should provide consistent, accurate results based on user input without errors or crashes.
    \item \textbf{Usability:} The user interface must be intuitive, with clear options for selecting drivers, cars, and venues.
\end{itemize}

\section{System Specifications}
The system specifications define the technical parameters under which the project will operate:
\begin{itemize}
    \item \textbf{Programming Language:} The F1 Simulation Project is implemented in C.
    \item \textbf{System Environment:} The system will be developed and tested on a Linux-based environment (Ubuntu) using the GCC compiler.
    \item \textbf{Hardware Requirements:} The system should run on a standard PC with at least 4GB RAM and 1GHz CPU.
\end{itemize}

\section{Conclusion}
The requirement engineering phase has provided a clear understanding of what the F1 Simulation Project must achieve. By focusing on both user and system requirements, the project aims to deliver a robust, user-friendly race simulation experience.

\newpage
